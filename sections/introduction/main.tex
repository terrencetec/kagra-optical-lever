\section{Introduction}
Optical lever (OpLev) is a device used to measure angular displacement of a reflective surface.
It consists of a light source (e.g. superluminescent LED), the reflective surface (e.g. the suspended main optics), and beam position sensing device (e.g. Quadphotodiode (QPD)).
In KAGRA, the beam used by the optical lever can also be used to measure the longitudinal displacement (along the reflective normal) of the reflective surface.
This is done by sensing the beam position behind a convex lens.
Although the phrase ``optical lever'' refers to the angular sensing part of the whole device, we refer the term ``optical lever'' in KAGRA to the whole device which senses all three displacements, longitudinal, pitch, and yaw.

There are two types of optical levers that are used as displacement sensors in KAGRA, regular and folded (the one used in MCo).
The regular optical lever system can be subdivided into two types, horizontal (e.g. those for Type-A and Type-Bp suspensions) and vertical (e.g. Those for Type-B suspensions).
In this document, we will focus mainly on the regular one and derive the sensing matrices for both horizontal and vertical configuration.
We will also discuss some misalignment of the optical lever, which leads to cross-coupling between different degrees of freedom in the sensing readout.
At last, we will provide some methods to obtain the sensing matrices.

This document is organized as follows.
In Sec.~\ref{sec:derivation}, the derivation of the sensing matrices for KAGRA's optical lever is provided.
In Sec.~\ref{sec:measuring_the_sensing_matrix_parameters}, we provided some ways to obtain the parameters in the sensing matrix that we have derived in Sec.~\ref{sec:derivation}.
