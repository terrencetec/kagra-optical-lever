\section{Derivation}
This section is organized as follows, we will first derive a very basic sensing matrix, assuming there's no misalignment in the optical system.
And then, we will derive the general sensing matrix that includes all sorts of misalignment.
Lastly, we will modify the matrix for a folded optical lever configuration.
\subsection{Very basic derivation}
In the simplest case, the beam position $x_1$ (along the incidence plane) is related to the angular displacement $\theta$ by
\begin{equation}
	x_1=\left(2r\right)\theta,
\end{equation}
where $r$ is the lever arm defined by the distance between the reflective surface and the sensing device.
The same beam can be used to measure the longitudinal displacement of the reflective surface, if the light beam has an angle of incidence $\alpha$.
In this case, the beam displacement reads
\begin{equation}
	x_1=\left(2r\right)\theta + \left(2\sin{\alpha}\right)x_L,
	\label{eqn:beam_displacement_l}
\end{equation}
where $x_L$ is the longitudinal displacement of the reflective surface.
As can be seen, equation.~\eqref{eqn:beam_displacement_l} shows a coupled sensor where it reads both the angular displacement and the longitudinal shift.
In KAGRA, some optical levers have a second sensor measuring the beam displacement $x_2$ some distance $d$ behind a convex lens with focal length $f$.
In this case, we obtain the second beam displacement $x_2$ via ray transfer matrices \cite{enwiki:1018856234}
\begin{equation}
	\begin{pmatrix}
		x_2\\
		\cdot
	\end{pmatrix}
	=
	\begin{bmatrix}
		1 & d\\
		0 & 1
	\end{bmatrix}
	\begin{bmatrix}
		1 & 0\\
		-1/f & 1
	\end{bmatrix}
	\begin{bmatrix}
		1 & r_\mathrm{lens} \\
		0 & 1
	\end{bmatrix}
	\begin{pmatrix}
		\left(2\sin{\alpha} \right)x_L\\
		2\theta
	\end{pmatrix},
\end{equation}
where $r_\mathrm{lens}$ is the distance between the reflective surface and the lens.
This gives
\begin{equation}
	x_2 = \left(2\sin\alpha\right)\left(1-\frac{d}{f}\right)x_L + 2\left[\left(1-\frac{d}{f}\right)r_\mathrm{lens}+d\right]\theta. 
	\label{eqn:beam_displacement_lens_1}
\end{equation}
Furthermore, we can place the second beam displacement sensor distance behind the lens.
So, if we set
\begin{equation}
	d=\frac{r_\mathrm{lens}f}{r_\mathrm{lens}-f},
	\label{eqn:d}
\end{equation}
then the angular coupling, i.e. the second term in Eqn.~\eqref{eqn:beam_displacement_lens_1}, becomes zero, effectively making the second beam displacement sensor a ``length'' (length as in longitudinal displacement) sensing device.
If Eqn.~\eqref{eqn:d} is satisfied, then the beam displacement measured by the second senor reads
\begin{equation}
	x_2 = \frac{-2f\sin\alpha}{r_\mathrm{lens}-f}x_L.
\end{equation}
Then, from here, we can diagonalize the sensors.
Consider a state-space model $\vec{y}=\mathbf{C}\vec{x}$, where $\vec{y}\equiv\begin{pmatrix}x_1\\x_2\end{pmatrix}$ are the measurements, and $\vec{x}\equiv\begin{pmatrix}x_L\\\theta\end{pmatrix}$ are the states.
Here, the $\mathbf{C}$ matrix is
\begin{equation}
	\mathbf{C}=
	\begin{bmatrix}
		2\sin\alpha & 2r\\
		\frac{-2f\sin\alpha}{r_\mathrm{lens}-f} & 0
	\end{bmatrix},
\end{equation}
which is not a diagonal matrix.
The goal is to define another measurement basis $\vec{y'}$ such that $\vec{y'}=\mathbf{C'}\vec{x}$, where $\mathbf{C'}$ is a diagonal matrix.
It's obvious that If we define $\vec{y'}\equiv\mathbf{C}^{-1}\vec{y}$, then $\mathbf{C'}$ becomes the identity, which is a diagonal matrix.
Therefore, we define $\mathbf{C}^\mathrm{-1}$ to be the sensing matrix, which maps sensor measurements to the displacements of the reflective surface.

\subsection{Misalignment}
