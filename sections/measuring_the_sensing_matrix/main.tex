\section{Measuring the sensing matrix parameters\label{sec:measuring_the_sensing_matrix_parameters}}
%As much as I would like to describe fully how we can obtain the sensing matrix, I cannot do so due to a restricted time frame.
%If there's enough demand\footnote{Spam my email: ttltsang@link.cuhk.edu.hk, or submit an issue to the Github repository of this document \url{https://www.github.com/terrencetec/kagra-optical-lever}.}, I will consider writing this section in full detail.
%Instead, I will lay down the idea and I am pretty sure you all can figure out how to do so.
Before going into the details, I would like to mention that measuring the parameters is \textbf{not} important.
It is not important anyway because all we need is the sensing matrix, but not the parameters.
And there're simpler ways to obtain the sensing matrix.
Also, we should say here, there's no way to obtain the parameters accurately.
It's impossible.
Instead, we are assuming that the misalignment parameters are small.
From here, we can roughly estimate the parameters in the sensing matrix.

If we want to use the sensing matrix \eqref{eqn:sensing_matrix_misaling_horizontal}, we need to find a few parameters.
Some are directly measurable or known, and some are not.
In particular, we need to find the lever arm from the optics to the tilt-sensing QPD $r$, the lever arm from the optics to the lens $r_\mathrm{lens}$, the angle of incidences $\alpha_h$ and $\alpha_v$, the position of the beam spot at the optics plane $\delta_x$ and $\delta_y$, the rotational angles $\phi_\mathrm{tilt}$ and $\phi_\mathrm{len}$, the focal length $f$, and the length-sensing QPD misplacement $\delta_d$.
Table~\ref{table:sensing_matrix_parameters} summarizes the parameters that we need to find. For a vertical setup, we know $\alpha_v$, but not $\alpha_h$.

\begin{table}[!h]
	\centering
	\begin{tabular}{|c|c|c|}
		\hline
		Parameter & Physical meaning & Measurable?\\
		\hline
		$r$ & lever arm from the optics to the tilt-sensing QPD plane & yes\\
		\hline
		$r_\mathrm{lens}$ & lever arm from the optics to the lens plane & yes\\
		\hline
		$\alpha_h$ & Angle of incidence of the beam projected on a horizontal plane & yes/no\\%\footnote{Roughly estimated fromt he CAD model. Closely related to the position of the viewports.}\\
		\hline
		$\alpha_v$ &  Angle of incidence of the beam projected on a vertical plane & no/yes\\
		\hline
		$\delta_x$ & Beam spot horizontal offset from the rotational center at the optics plane & no\\
		\hline
		$\delta_y$ & Beam spot vertical offset from the rotational center at the optics plane & no\\
		\hline
		$\phi_\mathrm{tilt}$ & Rotational angle between the ``yaw-pitch'' frame and the tilt-sensing QPD frame & no\\
		\hline
		$\phi_\mathrm{len}$ & Rotational angle between the ``yaw-pitch'' frame and the length-sensing QPD frame & no\\
		\hline
		$f$ & Focal length of the convex lens & known\\
		\hline
		$\delta_d$ & Misplacement of the length-sensing QPD from the $d=\frac{r_\mathrm{lens}f}{r_\mathrm{lens}-f}$ point & no\\
		\hline
	\end{tabular}
	\caption{Parameters of the sensing matrix for a horizontal/vertical optical lever layout.}
	\label{table:sensing_matrix_parameters}	
\end{table}

\subsection{Measuring the rotational angles $\phi_\mathrm{tilt}$ and $\phi_\mathrm{len}$}
First we need to align the QPD axes to the desired frame, i.e. finding $\phi_\mathrm{tilt}$ and $\phi_\mathrm{len}$.
For $\phi_\mathrm{tilt}$, we need to identify resonance frequencies of the pure pitch and pure yaw resonances, $f_\mathrm{pitch}$ and $f_\mathrm{yaw}$.
Then, we excite the optics using the coil magnet actuators.
Note here, we must excite the optics with white noise only.
The best way is to ``kick'' the suspensions and measure the free swing spectra.
However, this is not always possible because the resonances might have very low Q-value and so the amplitude decays very quickly to a level lower than the sensor noise.
We cannot use a single frequency line injection, or else we cannot distinguish cross-coupling between from the sensors and from the actuators.

If we excite the optics, we will see two peaks at resonances frequencies $f_\mathrm{pitch}$ and $f_\mathrm{yaw}$ in both $x_\mathrm{tilt}$ and $y_\mathrm{tilt}$ spectra.
We can pick one of them.
If we pick $f_\mathrm{pitch}$, then we can calculate the angle by
\begin{equation}
	\tan\phi_\mathrm{tilt} = -\frac{x_\mathrm{tilt}\mleft(f_\mathrm{pitch}\mright)}{y_\mathrm{tilt}\mleft(f_\mathrm{pitch}\mright)},
	\label{eqn:phi_tilt_from_pitch}
\end{equation}
where $x_\mathrm{tilt}\mleft(f_\mathrm{pitch}\mright)$ is the amplitude spectral density at frequency $f_\mathrm{pitch}$, and $y_\mathrm{tilt}\mleft(f_\mathrm{pitch}\mright)$ is the amplitude spectral density at frequency $f_\mathrm{pitch}$.
Note here, $\frac{x_\mathrm{tilt}\mleft(f_\mathrm{pitch}\mright)}{y_\mathrm{tilt}\mleft(f_\mathrm{pitch}\mright)}$ is in fact a transfer function and has relative phase.
In theory, the phase should be either $0^\circ$ or $\pm180^\circ$.
We need to add an additional minus sign if the phase is at $\pm180^\circ$.

If we choose $f_\mathrm{yaw}$, then we can calculate the angle by
\begin{equation}
	\tan\phi_\mathrm{tilt} = \frac{y_\mathrm{tilt}\mleft(f_\mathrm{yaw}\mright)}{x_\mathrm{tilt}\mleft(f_\mathrm{yaw}\mright)}.
	\label{eqn:phi_tilt_from_yaw}
\end{equation}
It's normal that the resulting angle calculated by Eqn.~\eqref{eqn:phi_tilt_from_pitch} and Eqn.~\eqref{eqn:phi_tilt_from_yaw} are not matched.
This can be explained by non-negligible $\delta_x$ and $\delta_y$, which tilts the pure yaw and pure pitch axes asymmetrically.
If they differ by a lot, then this method automatically fails here.
If they are similar, then let's just pick one or simply pick the average, or even somewhere randomly between.
%But this is not usually the case.
%That's why there's a motivation for having separate rotation angles horizontal and vertical QPD axes.
%But we don't discuss further here.

Now, for the angle $\phi_\mathrm{len}$, assuming small misalignment, let's excite the longitudinal mode at resonance frequency $f_\mathrm{longitudinal}$.
Then, the angle can be calculated using
\begin{equation}
	\tan\phi_\mathrm{len} = \frac{x_\mathrm{len}\mleft(f_\mathrm{longitudinal}\mright)}{y_\mathrm{len}\mleft(f_\mathrm{longitudinal}\mright)}.
\end{equation}
Again, note the phase between $x_\mathrm{len}$ and $y_\mathrm{len}$.
For a vertical setup, we can calculate the angle using
\begin{equation}
	\tan\phi_\mathrm{len} = -\frac{y_\mathrm{len}\mleft(f_\mathrm{longitudinal}\mright)}{x_\mathrm{len}\mleft(f_\mathrm{longitudinal}\mright)}.
\end{equation}

\subsection{Measuring misplacement of the length-sensing QPD $\delta_d$ \label{sec:measuring_misplacement}}
Now, for a horizontal optical lever, with the rotational angles obtained, we can read $x_\mathrm{len}'=\cos\phi_\mathrm{len}x_\mathrm{len}+\sin\phi_\mathrm{len}y_\mathrm{len}$.
Using Eqn.~\eqref{eqn:x_len_misplaced_qpd}, we see that, without a yaw motion, the misplacement $\delta_d$ can be found by
\begin{equation}
	\delta_d = -\left(\frac{x_\mathrm{len}'}{x_L}+\frac{2f\sin\alpha_h}{r_\mathrm{lens}-f}\right)\frac{f}{2\sin\alpha_h}.
\end{equation}
And, for a vertical setup, assuming no pitch, we get
\begin{equation}
	\delta_d = -\left(\frac{y_\mathrm{len}'}{x_L}+\frac{2f\sin\alpha_v}{r_\mathrm{lens}-f}\right)\frac{f}{2\sin\alpha_v}.
\end{equation}
	
Here, we assume that $r_\mathrm{lens}$, $\alpha_h$ (for horizontal setup), $\alpha_v$ (for vertical setup), and $f$ are known.
The problem is how to get $x_L$.
For suspensions with an upper stage, such as preisolator, we can offset the whole suspension in longitudinal direction using the actuators at upper stages, and we can measure $x_L$ at that stage.
Doing so, we assume that the sensors at the upper stage are diagonalized and are better calibrated.
In this sense, this approach is better for the vertical setup because a preisolator can never offset the suspension in pitch!

\subsection{Measuring angle of incidences $\alpha_h$ and $\alpha_v$}
For a horizontal setup, $\alpha_h$ is known, and we need to measure $\alpha_v$.
To do so, we can use the same trick used in Sec.~\ref{sec:measuring_misplacement} and offset the whole suspension chain in longitudinal direction.
Using Eqn.~\eqref{eqn:y_tilt'}, we can straightforwardly write down the angle $\alpha_v$, assuming no pitch,
\begin{equation}
	\sin\alpha_v = \frac{y_\mathrm{tilt}'}{2x_L}.
\end{equation}
For a vertical setup, assuming no yaw,
\begin{equation}
	\sin\alpha_h = \frac{x_\mathrm{tilt}'}{2x_L}.
\end{equation}
Again, $x_L$ here is estimated from a sensor at the upper stages, which are diagonalized and calibrated.

\subsection{Measuring the beam offsets $\delta_x$ and $\delta_y$}
Now, we have all the parameters except the beam offsets $\delta_x$ and $\delta_y$.
So, let's plug all the parameters into Eqn.~\eqref{eqn:sensing_matrix_misaling_horizontal} and Eqn.~\eqref{eqn:sensing_matrix_misaling_vertical} to get some temporary longitudinal readout $\left(x_{L,\mathrm{temp}},\, \theta_{P,\mathrm{temp}},\, \theta_{Y,\mathrm{temp}}\right)$.
Then, we can excite the pitch and yaw resonances at $f_\mathrm{pitch}$ and $f_\mathrm{yaw}$, and measure the spectra using the temporary readouts.
The offsets can be obtained straightforwardly by
\begin{equation}
	\delta_x = \frac{x_{L,\mathrm{temp}}\mleft(f_\mathrm{yaw}\mright)}{\theta_{Y,\mathrm{temp}}\mleft(f_\mathrm{yaw}\mright)},
\end{equation}
and
\begin{equation}
	\delta_y = \frac{x_{L,\mathrm{temp}}\mleft(f_\mathrm{pitch}\mright)}{\theta_{P,\mathrm{temp}}\mleft(f_\mathrm{pitch}\mright)}.
\end{equation}
And this concludes all the measurements needed to obtain the whole sensing matrix.
