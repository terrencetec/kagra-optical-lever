\section{Conclusion}
We have derived the general sensing matrix \eqref{eqn:sensing_matrix_general} for an optical lever that has an incidence plane tilted arbitrarily compared to the horizontal plane or the vertical plane.
In KAGRA, the incidence plane is either roughly on the horizontal plane or the vertical plane.
We have reduced the general sensing matrix down for horizontal optical levers and vertical optical levers as initial sensing matrice for KAGRA, Eqn.~\eqref{eqn:sensing_matrix_horizontal}, and Eqn.~\eqref{eqn:sensing_matrix_vertical}, respectively.
We have discussed three misalignment mechanisms, namely, rotation of the QPD frame, miscentering of the QPD beam at the optics plane, and misplacement of the length-sensing QPD.
And, we have modified the general sensing matrix to account for these types of misalignement.
The most general sensing matrix is written in Eqn.~\eqref{eqn:sensing_matrix_general_misalignment}.
Again, it's reduced down to a horizontal configuration and a vertical configuration, Eqn.~\eqref{eqn:sensing_matrix_misaling_horizontal} and Eqn.~\eqref{eqn:sensing_matrix_misaling_horizontal}, respectively.
At last, we have discussed some possible methods to obtain the parameters in the sensing matrix, and hence the sensing matrix.
We note that this is neither the easiest nor the best way, to obtain the sensing matrix.
There are faster and better ways to do that, e.g. measuring coupling ratios and compute the inverse directly.